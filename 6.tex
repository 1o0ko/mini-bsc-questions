\chapter{Twierdzenia i wzory całkowe Cauchy'ego.}

\section{Podstawowe informacje}

\subsection{Funkcje holomorficzne}

\begin{df}{Funkcja holomorficzna}
Mówimy, że funkcja $f:\mathbb{C}\rightarrow \mathbb{C}$ jest \textit{holomorficzna w punkcie} $z\in \mathbb{C}$, jeżeli jest ona określona w pewnym otoczeniu tego punktu i ma pochodną w każdym punkcie tego otoczenia. Mówimy, że funkcja $f$ jest \textit{holomorficzna w zbiorze} $D\subseteq \mathbb{C}$ jeżeli jest holomorficzna w każdym punkcie tego zbioru.
\end{df}

\begin{df}{Równania Cauchy'ego - Riemanna}
Funkcja $f:D\rightarrow\mathbb{C}$ której rozkład na część rzeczywistą i urojoną jest postaci $f(x+iy)=u(x,y)+iv(x,y)$ jest holomorficzna w $D$ $\Leftrightarrow$ gdy $u$ i $v$ są różniczkowalne i spełniają równania różniczkowe Cauchy'ego - Riemanna:
\begin{displaymath}
\left\{\begin{array}{ll}
\ \ \displaystyle{u_x=v_y}
\\
\\\displaystyle{-u_y=v_x}
\end{array} \right.
\end{displaymath}
Wtedy $f'(z)=u_x+iv_x=v_y-iu_y=u_x-iu_y=v_y+iv_x$
\end{df}

\begin{przyk}{(Funkcja kwadratowa)}\\
$f(z)=z^2=(x^2-y^2)+2xyi\rightarrow u_x=2x=v_y,\ -u_y=2y=v_x$
\end{przyk}

\begin{przyk}{(Inwersja)}\\
$f(z)=\dfrac{1}{z}=\dfrac{x}{x^2+y^2}-\dfrac{iy}{x^2+y^2}\rightarrow u_x=\frac{y^2-x^2}{(x^2+y^2)^2}=v_y$
$-u_y=\frac{2xy}{(x^2+y^2)^2}=v_x$
\end{przyk}

\begin{przyk}{(Funkcja wykładnicza)}\\
$f(z)=e^z=e^x cos(y)+ie^x sin(y)\rightarrow u_x=e^x cos(y)=v_y,\ -u_y=e^x sin(y)=v_x$
\end{przyk}

\subsection{Twierdzenie podstawowe Cauchy'ego}

\begin{df}{Twierdzenie całkowe Cauchy'ego}
Niech $G\subseteq \mathbb{C}$ będzie obszarem jednospójnym, $f:G\rightarrow\mathbb{C}$ będzie holomorficzna, oraz niech $\gamma$ będzie kawałkami gładką krzywą zamkniętą leżącą w obszarze $G$. Wówczas:
$$\oint\limits_{\gamma}f(z)dz=0$$
\end{df}

\begin{uwg}
Twierdzenia tego używa się w dowodach wielu twierdzeń z analizy zespolonej m.in. twierdzenia o residuach czy wzoru całkowego Cauchy'ego
\end{uwg}

\subsection{Punkty regularne i osobliwe}

\begin{df}{(Punkt regularny)}\\
Mówimy, że $z_0$ jest \textbf{punktem regularnym} funkcji $f(z)$ jeśli jest ona holomorficzna w tym punkcie.
\end{df}

\begin{df}{(Punkt osobliwy)}\\
Punkt $z_0$ nazywamy \textbf{punktem osobliwym, odosobnionym (izolowanym)} jeśli istnieje sąsiedztwo $0<|z-z_0|<R$ tego punktu w którym funkcja $f(z)$ jest holomorficzna.
\end{df}

\begin{przyk}
$f(z)=\frac{1}{1-z}$ punkt $z=1$ jest punktem osobliwym odosobnionym, pozostałe punkty $\mathbb{C}$ są regularne
\end{przyk}

\section{Twierdzenia}

\subsection{Twierdzenie o residuach}

\begin{df}{(Definicja residuum)}\\
Niech $f$ holomorficzna w $G_{z_0}=\lbrace z:\ 0<|z-z_0|<\varepsilon\rbrace$ i $\gamma$ dowolny dodatnio skierowany okrąg $|z-z_0|=r<\varepsilon$. Wówczas $$a_{-1}=\frac{1}{2\pi i} \oint\limits_{\gamma} f(z)\ dz$$ gdzie $a_{-1}$ jest współczynnikiem szeregu Laurenta postaci $$f(z)=\sum_{n=-\infty}^{\infty} a_n z^n$$ Liczbę $a_{-1}$ nazywamy \textbf{residuum} funkcji $f(z)$ w punkcie $z_0$ i oznaczmy przez $Res_{z_0} f(z)$. 
\end{df}

\begin{tw}
Jeżeli $f:G\rightarrow\mathbb{C}$ jest holomorficzna w obszarze $G\subseteq \mathbb{C}$ z wyjątkiem co najwyżej sko{\'n}czonej liczby punktów $z_1,\ z_2, \ldots,\ z_n \in G$ zaś $\gamma$ jest krzywą zamkniętą, kawałkami gładką, dodatnio zorientowaną, leżącą w tym obszarze oraz zawierającą wskazane punkty w swoim wnętrzu to:
$$\oint\limits_{\gamma}f(z)dz=2\pi i \sum_{k=1}^{n}\bigg(\nu_{\gamma}(z_k)\cdot Res_{z_k}f \bigg)$$ gdzie $\nu_{\gamma}(z_k)$ to indeks punktu $z_0$ względem krzywej $\gamma$ (intuicyjnie jest to ilość okrążeń krzywej $\gamma$ dookoła punktu $z_0$).
\end{tw}

\begin{tw}{Jak liczyć residua z funkcji holomorficznych w $G\setminus\lbrace z_0,\ldots, z_n\rbrace$ ?}
Niech f będzie holomorficzna w $G$ oraz $z_0\in G$ wówczas: $$Res_{z_0}\bigg(\dfrac{f(z)}{z-z_0}\bigg)=f(z_0)$$
\end{tw}

\begin{przyk}
Obliczyć całkę $\int\limits_{C} \frac{dz}{z^2(z+2i)}$ gdzie $C$ jest okręgiem $|z+2i|=1$ dodatnio zorientowanym $$\oint\limits_{|z+2i|=1}\frac{dz}{z^2(z+2i)}=2\pi i \cdot \nu_{\gamma}(-2i)\cdot Res_{-2i}\bigg( \frac{1}{z^2(z+2i)}\bigg)$$ $$\oint\limits_{|z+2i|=1}\frac{dz}{z^2(z+2i)}=2\pi i \cdot 1 \cdot \frac{1}{(-2i)^2}=-\frac{\pi i}{2}$$
\end{przyk}

\begin{przyk}
$$\oint\limits_{|z|=3}\frac{e^z}{z^2+2z}\ dz=\frac{1}{2}\oint\limits_{|z|=3}\frac{e^z}{z}\ dz-\frac{1}{2}\oint\limits_{|z|=3}\frac{e^z}{z+2}\ dz=$$ $$=2\pi i \cdot \bigg[\frac{1}{2} Res_0 \bigg(\frac{e^z}{z}\bigg)-\frac{1}{2} Res_{-2} \bigg(\frac{e^z}{z-(-2)}\bigg)\bigg]=2\pi i \cdot \bigg(\frac{1}{2}-\frac{e^{-2}}{2}\bigg)$$ zatem ostatecznie mamy: $$\oint\limits_{|z|=3}\frac{e^z}{z^2+2z}\ dz=\pi i \cdot \big(1-e^{-2}\big)$$
\end{przyk}

\begin{df}{(Klasyfikacja punktów osobliwych)}\\
Niech dane będzie rozwinięcie funkcji $f(z)$ w szereg Laurenta w sąsiedztwie punktu osobliwego, odizolowanego $z_0$: $$f(z)=\sum_{n=-\infty}^{\infty} a_n z^n=\sum_{n=0}^{\infty} a_n (z-z_0)^n+\sum_{n=1}^{\infty} \frac{a_{-n}}{(z-z_0)^n}$$ 
\begin{itemize}
  \item
    Jeśli część osobliwa rozwinięcia ''znika'' to punkt $z_0$ nazywamy \textbf{pozornie osobliwym}.
  \item
    Jeśli część osobliwa zawiera skończoną liczbę składników to istnieje takie $k\in \mathbb{N}$, że $a_{-k}\neq 0$ oraz $a_{-n}=0$ dla $n>k$ wtedy $z_0$ nazywamy \textbf{biegunem k-krotnym}
  \item
    Jeśli część osobliwa zawiera nieskończenie wiele składników to punkt $z_0$ nazywamy \textbf{istotnie osobliwym}.
\end{itemize}
\end{df}

\begin{tw}{(Residuum w biegunie k-krotnym)}\\
Jeśli $z_0$ jest biegunem k-krotnym funkcji $f(z)$ to residuum w punkcie $z_0$ funkcji $f(z)$ wynosi: $$Res_{z_0}f(z)=\frac{1}{(k-1)!}\lim_{z \to z_0}\bigg[\frac{d^{k-1}}{dz^{k-1}}\bigg((z-z_0)^k f(z)\bigg)\bigg]$$
\end{tw}

\begin{tw}{(Odpowiedniość między biegunami a zerami funkcji)}\\
\begin{itemize}
  \item
    Jeśli $z_0$ jest biegunem k-krotnym funkcji $f(z)$ to dla funkcji:
	\begin{displaymath}
	\left\{\begin{array}{ll}
	\ \ \displaystyle{\frac{1}{f(z)},\quad z\neq z_0}
	\\
	\\\displaystyle{\qquad 0,\quad z=z_0}
	\end{array} \right.
	\end{displaymath}
	jest on k-krotnym zerem.
  \item
    Jeśli $z_0$ jest k-krotnym zerem funkcji $f(z)$ to jest on biegunem k-krotnym funkcji $\frac{1}{f(z)}$ (bardzo pomocne przy liczeniu residuów w biegunach)
\end{itemize}
\end{tw}

\subsection{Wzór całkowy Cauchy'ego}

\begin{tw}
Niech $f:G\rightarrow\mathbb{C}$ będzie holomorficzna w obszarze $G\subseteq \mathbb{C}$ oraz niech $\{z: |z-z_0|\leq r\}\subset G$. Wtedy dla każdego $z$ takiego, że $|z-z_0|<r$ mamy $$f(z)=\frac{1}{2\pi i}\oint\limits_{|w-z_0|=r}\frac{f(w)}{w-z}\ dw$$ zatem funkcja holomorficzna zdefiniowana na dysku jest całkowicie zdeterminowana przez wartości, które przyjmuje na brzegu tego dysku.
\end{tw}

\begin{przyk}
$$\oint\limits_{|z|=2}\frac{z^2}{z^2+2z+2}\ dz=\oint\limits_{C_1}\frac{\frac{z^2}{z-z_2}}{z-z_1}\ dz+\oint\limits_{C_2}\frac{\frac{z^2}{z-z_1}}{z-z_2}\ dz=$$ $$=2\pi i \bigg(\frac{z_{1}^2}{z_1-z_2}+\frac{z_{2}^2}{z_2-z_1}\bigg)=2\pi i (-2)=-4\pi i$$
gdzie $z_1=-1+i,\ z_2=-1-i$ są punktami osobliwymi funkcji $\frac{z^2}{z^2+2z+2}$
\end{przyk}

\subsection{Inne zastosowania}

\begin{tw}
Niech funkcja $f(z)=\frac{p(z)}{q(z)}$ będzie zespoloną funkcją wymierną gdzie $p(z)$ i $q(z)$ są wielomianami o współczynikach rzeczywistych, ponadto niech $\forall x\in \mathbb{R},\ q(x)\neq 0$ oraz $deg\ q\geq 2+deg\ p$ wtedy: $$\int\limits_{-\infty}^{\infty}f(x)\ dx=2\pi i\cdot\Bigg[\sum_{Im (z_k)>0}\bigg(Res_{z_k}f \bigg)\Bigg]$$
\end{tw}

\begin{przyk}
$$\int\limits_{-\infty}^{\infty}\frac{dx}{1+x^2}=2\pi i\cdot\Bigg[\sum_{Im (z_k)>0}\bigg(Res_{z_k}\bigg(\frac{1}{1+z^2}\bigg) \bigg)\Bigg]$$
Stąd miejsca zerowe $1+z^2$ (a zatem bieguny $\frac{1}{1+z^2}$) wynoszą $z_1=i,\ z_2=-i$ zaś residuum wynosi $$Res_{i}\bigg(\frac{1}{1+z^2}\bigg)=Res_{i}\bigg(\frac{1}{(z+i)(z-i)}\bigg)=\frac{1}{i+i}=\frac{1}{2i}$$ więc w ''prosty'' sposób otrzymujemy znany wzór: $$\int\limits_{-\infty}^{\infty}\frac{dx}{1+x^2}=2\pi i\cdot \frac{1}{2i}=\pi$$
\end{przyk}