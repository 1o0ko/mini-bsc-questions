\chapter{Miara Lebesgue’a, funkcje~mierzalne, całka~Lebesgue’a, tw.~Fubiniego}
	\begin{df}
		Niech $\Omega$ będzie dowolnym niepustym zbiorem i niech $\mathcal{F}$ będzie niepustą rodziną podzbiorów zbioru $\Omega$. $\mathcal{F}$ nazywamy\textbf{ $\sigma$-ciałem podzbiorów zbioru }$\Omega$, jeżeli spełnia następujące warunki:
		\begin{enumerate}
			\item $\emptyset \in \mathcal{F}$, gdzie symbolem $\emptyset$ oznaczamy zbiór pusty,
			\item jeżeli $A \in \mathcal{F}$, to $\Omega \setminus A \in \mathcal{F}$,
			\item jeżeli $A_1, A_2, \ldots \in \mathcal{F}$, to $\cup_{n=1}^{\infty} \in \mathcal{F}$.
		\end{enumerate}
	\end{df}	
	
	\begin{przyk}
		Zbiór wszystkich podzbiorów zbioru $X$ jest $\sigma$-algebrą. W tym przypadku stosujemy oznaczenie $\mathcal{A} = 2^X$.
	\end{przyk}
	
	\begin{przyk}
		$\mathcal{A} = \{\emptyset, X\}$ jest $\sigma$-algebrą, zwaną \textit{$\sigma$-algebrą trywialną.}
	\end{przyk}
	
	\begin{tw}
		Niech $\Omega$ będzie dowolnym niepustym zbiorem i niech $\mathcal{A}$ będzie niepustą rodziną podzbiorów zbioru $\Omega$, wtedy istnieje najmniejsze $\sigma$-ciało podzbiorów zbioru $\Omega$ zawierający $\mathcal{A}$. Takie $\sigma$-ciało nazywać będziemy $\sigma$-ciałem generowanym przez rodzinę $\mathcal{A}$.
	\end{tw}
	
	\begin{df}
		Niech $(X, \rho)$ będzie przestrzenią metryczną, a $\Gamma$ będzie rodziną zbiorów otwartych (lub domkniętych). Wtedy $\sigma$-algebrę generowana przez rodzinę $\Gamma$, oznaczaną  $\mathcal{B}(X)$,  nazywamy \textit{$\sigma$-algebrą zbiorów borelowskich}, a jej elementy \textit{zbiorami borelowskimi}.
	\end{df}
	
	\begin{df}
		Niech $X$ będzie niepustym zbiorem, a $\mathcal{A}$ ustaloną $\sigma$-algebrą podzbiorów zbioru $X$. Funkcję $\mu:\mathcal{A} \rightarrow [0, \infty]$ spełniającą następujące warunki
			\begin{enumerate}
				\item $\mu(\emptyset) = 0$,
				\item dla dowolnego ciągu $(A_k)_{k \in \mathbb{N}}$ zbiorów z $\mathcal{A}$ takiego, że $A_i \cap A_j = \emptyset$, dla $i \neq j$, zachodzi równość
				\begin{equation}
				\mu \left( \bigcup_{k=1}^{\infty} A_k \right) = \sum_{k=1}^{\infty} \mu(A_k), 
				\end{equation}
			\end{enumerate}
		nazywamy \textit{miarą}. 
	\end{df}
	
	\begin{przyk}
		Niech $X$ będzie dowolnym zbiorem i $\mathcal{A} = 2^X$. Miarę $\mu$ na $\mathcal{A}$ definiujemy następująco:		
		\[
		\mu(A) =
		\begin{cases}
			|A|, &\text{jeżeli $A$ jest zbiorem skończonym}, \\
			\infty, & \text{w przeciwnym przypadku}.
			\end{cases}
		\]
		Taką miarę nazywamy \textit{miarą liczącą}.
	\end{przyk}
	
	\begin{tw}\label{lebesque_measure}
		Niech $X = \textbf{R}^n$ i niech $\mathcal{A}$ będzie $\sigma$-algebrą zbiorów borelowskich w $X$. Wtedy istnieje dokładnie jedna miara $\lambda_n$ określona na $\mathcal{A}$ taka, że dla dowolnego prostokąta 
		$$
			P = [a_1,b_1]\times \ldots \times [a_n, b_n]
		$$
		mamy 
		$$
		\lambda_n(P) = (b_1 - a_1) \cdot \ldots \cdot (b_n - a_n).
		$$
	\end{tw}
	
	\begin{df}
		Miarę zdefiniowaną w twierdzeniu \ref{lebesque_measure} nazywamy miarą \textit{Lebesque'a na zbiorach borelowskich}.
	\end{df}
	
	\begin{df}
		Parę $(X, \mathcal{A})$ taką, że $\mathcal{A}$ jest $\sigma$-ciałem podzbiorów zbioru $X$ nazywać będziemy \textit{przestrzenią mierzalną}.
	\end{df}
	
	\begin{df}
		Trójkę $(X, \mathcal{A}, \mu)$ taką, że $\mathcal{A}$ jest $\sigma$-ciałem podzbiorów zbioru $X$, a $\mu$ miarą określoną na $\mathcal{A}$ nazywać będziemy \textit{przestrzenią mierzalną z miarą}.
	\end{df}
	
	\begin{df}
		Niech  $(X, \mathcal{A}, \mu)$ przestrzenią mierzalną z miarą. Ciąg zbiorów mierzalnych $(A_n)_{n \in \mathbb{N}}$ nazywamy \textit{wstępującym}, gdy $A_n \subset A_{n+1}$ dla $n \in \mathbb{N}$, \textit{zstępującym}, gdy  $A_{n+1} \subset A_{n}$ dla $n \in \mathbb{N}$.  
	\end{df}
	
	\begin{tw}[Ciągłość miary]
		Jeżeli $(A_n)_{n \in \mathbb{N}}$ jest wstępującym ciągiem zbiorów mierzalnych, to 
		\begin{equation*}
			\mu \left( \bigcup_{n=1}^{\infty} A_n \right) = \lim_{n \rightarrow \infty } \mu(A_n).
		\end{equation*}

		Jeżeli $(A_n)_{n \in \mathbb{N}}$ jest zstępującym ciągiem zbiorów mierzalnych i $\mu(A_1) < \infty $, to 
		\begin{equation*}
		\mu \left( \bigcap_{n=1}^{\infty} A_n \right) = \lim_{n \rightarrow \infty } \mu(A_n).
		\end{equation*}
	\end{tw}
	
	\section{Funkcje mierzalne}
	Poprzez $\bar{\mathbb{R}}$ będziemy oznaczali zbiór liczb rzeczywistych uzupełniony o dwa elementy: $-\infty, +\infty$.
	\begin{df}
		Niech $(X, \mathcal{F})$ oraz $(Y, \mathcal{E})$ będą dwoma przestrzeniami mierzalnymi. Odwzorowanie $f:  X \rightarrow Y$ nazywamy $(\mathcal{F}, \mathcal{E})$-mierzalnym, jeżeli przeciwobraz każdego zbioru $E \in \mathcal{E}$ względem $f$ należy do $\mathcal{F}$, tj.
		\begin{equation*}
			f^{-1}(E) := \{ x \in X \: | \: f(x) \in E\} \in \mathcal{F}, \quad \forall E \in \mathcal{E}.
		\end{equation*}
	\end{df}		
	
	\begin{uwg}
		W dalszej części wywodu mówić będziemy o funkcjach, dla których przestrzeń mierzalna $(Y, \mathcal{E})$ jest postaci 	$(\bar{\mathbb{R}}, \mathcal{B}(\mathbb{R}))$, gdzie $\mathbb{R}$ oznacza rozszerzony zbiór liczb rzeczywistych, a $\mathcal{B}(\bar{\mathbb{R}})$  $\sigma$-ciało zbiorów borelowskich.
	\end{uwg}
	
	\begin{tw}
		Jeżeli $X$ jest przestrzenią metryczną, a $\mathcal{A}$ jest $\sigma$-algebrą zbiorów borelowskich, to dowolna funkcja ciągła $f: X \rightarrow \mathbb{R}$ jest mierzalna.
	\end{tw}
		
	\section{Całka  Lebesgue’a}
	\subsection{Całka z funkcji charakterystycznej zbioru}
	\begin{df}
	\textit{Funkcją charakterystyczną} zbioru $A \subset X$ nazywamy funkcję $\chi_{A} : X \rightarrow \mathbb{R}$ określoną wzorem:		
		\begin{equation}
			\chi_A(x) = 
			\begin{cases}
				1, &\text{dla $x \in A$}, \\
				0, &\text{dla $x \not \in A$}.
			\end{cases}
		\end{equation}
	\end{df}
	
	\begin{df}
		Niech $\chi_{A}$ będzie funkcją charakterystyczną  zbioru $A \subset X$. Wtedy całkę funkcji $\chi_{A}$ względem miary $\mu$ definiujemy jako:
		\begin{equation}
		\int_X \chi_{A}(x) d \mu = \mu(A).
		\end{equation}
	\end{df}
	
	\subsection{Całka z funkcji prostej}
	\begin{df}
		\textit{Funkcją prostą} nazywamy funkcję o skończonym zbiorze wartości. 
	\end{df}
	\begin{uwg}
		Każdą funkcję prostą $f$ można przedstawić jako kombinację liniową funkcji charakterystycznych:
		\begin{equation}
			f(x) = \sum_{i=1}^n a_i \chi_{A_i}(x), \quad \text{gdzie } A_i = \{ x \in X : f(x) =  a_i \}.
		\end{equation}
	\end{uwg}	
	
	\begin{df}
		Niech $f_n$ będzie funkcją prosta, nieujemną i mierzalną określoną na zbiorze X. Wtedy całką funkcji $f_n$ względem miary $\mu$ definiujemy jako:				
		\begin{equation}
			 \int_X f(x) d \mu= \sum_{i=1}^n a_i \mu(A_i).
		\end{equation}
	\end{df}
	
	\subsection{Całka z nieujemnej funkcji mierzalnej}
	\begin{df}
		Niech $f$ będzie nieujemną funkcją mierzalną, a $S$ rodziną funkcji prostych mierzalnych. Wtedy całkę funkcji $f$ względem miary $\mu$ definiujemy jako:
		\begin{equation}
			\int_X f \, d\mu = \sup_{s \in S} \left\{ \int_X s \, d\mu: 0 \leq s \leq f \right\}.
		\end{equation}
	\end{df}

	\begin{tw}
		Niech $f$ będzie nieujemną funkcją mierzalną. Wtedy istnieje niemalejący ciąg $f_n$ funkcji prostych, nieujemnych i mierzalnych taki, że
		$$
		\forall_{x \in X} \lim_{x \rightarrow \infty} f_n(x) = f(x).
		$$
	\end{tw}
	
	\begin{tw}
		Niech $f$ będzie nieujemną funkcją mierzalną, a $f_n$ ciągiem nieujemnych mierzalnych funkcji prostych zbieżnych punktowo do $f$. Wtedy: 
		\begin{equation}
				\int_X f \, d\mu = \lim_{n \rightarrow \infty} \int_X f_n \, d\mu.
		\end{equation}
		
	\end{tw}
	\subsection{Całka z funkcji mierzalnej}
	\begin{df}
		Mówimy, że funkcja mierzalna $f:X \rightarrow \mathbb{R}$ jest całkowalna sensie Lebesque'a, jeżeli
		\begin{equation}
			\int_X |f| \, d \mu < \infty.
		\end{equation}
		Wtedy całkę funkcji $f$ względem miary $\mu$ definiujemy jako:
		\begin{equation}
			\int_X f \, d\mu = \int_X f^+ \, d\mu -  \int_X f^- \, d\mu, 
		\end{equation}
		gdzie $f = f^+ - f^-$ jest rozkładem funkcji $f$ na $f^+ = \max(f, 0)$ oraz $f^- = -\min(f, 0)$.
	\end{df}
	
	\begin{tw}
		Niech $f: [a,b] \rightarrow \mathbb{R}$ będzie ograniczoną funkcją całkowalną w sensie Riemanna. Wtedy $f$ jest $\lambda$-mierzalna i obie całki są równe:
		\begin{equation}
			\int_a^b f(x) dx = \int_{[a,b]}f \, d\lambda.
		\end{equation}
	\end{tw}
	\section{Twierdzenie Fubiniego}
	
	\begin{tw}
		Niech dane będą dwie przestrzenie mierzalne z miarami $(X_1, \mathcal{A}_1, \mu_1)$ oraz $(X_2, \mathcal{A}_2, \mu_2)$. Najmniejsze $\sigma$-ciało zawierające rodzinę wszystkich zbiorów postaci $A_1 \times A_2$, gdzie $A_1 \in \mathcal{A}_1, A_2 \in \mathcal{A}_2$ nazywamy \textit{$\sigma$-ciałem produktowym} $\sigma$-ciał $\mathcal{A}_1$ i $\mathcal{A}_2$ i  oznaczamy poprzez $\mathcal{A}_1 \otimes \mathcal{A}_2$.
	\end{tw}
	
	\begin{tw}
		Niech $(X_1, \mathcal{A}_1, \mu_1)$ oraz $(X_2, \mathcal{A}_2, \mu_2)$ będą dwoma $\sigma$-skończonymi przestrzeniami mierzalnymi z miarą. Na $\sigma$-ciele $\mathcal{A}_1 \otimes \mathcal{A}_2$ istnieje tylko jedna miara $\mu_1 \otimes \mu_2$ spełniająca dla każdego $A_1 \in \mathcal{A}_1$ oraz  $A_2 \in \mathcal{A}_2$ warunek
		\begin{equation}
			\mu_1 \otimes \mu_2(A_1 \times A_2) = \mu_1(A_1) \cdot \mu_2(A_2).
		\end{equation}
	\end{tw}
	
	\begin{tw}
		Niech $(X_1, \mathcal{A}_1, \mu_1)$ oraz $(X_2, \mathcal{A}_2, \mu_2)$ będą dwoma $\sigma$-skończonymi przestrzeniami mierzalnymi z miarą. Niech funkcja $f: X_1 \times X_2 \rightarrow \mathbb{R}$ będzie $\mathcal{A}_1 \otimes \mathcal{A}_2$ mierzalna, oraz $\mu_1 \otimes \mu_2$ całkowalna, wtedy funkcje:
		\begin{equation*}
			I: x_1 \mapsto \int_{X_2} f(x_1,x_2) d\mu_2(x_2), \quad J: x_2 \mapsto \int_{X_1} f(x_1,x_2) d\mu_1(x_1), 
		\end{equation*}
		są odpowiednio $\mathcal{A}_1$ oraz $\mathcal{A}_2$ mierzalne, oraz:
		\begin{align*}
			\int_{X_1 \times X_2} f \, d \mu_1 \otimes \mu_2 
			 &=  \int_{X_1} \left(  \int_{X_2} f(x_1,x_2) d\mu_2(x_2) \right) d\mu_1(x_1) \\
			 &=  \int_{X_2} \left(  \int_{X_1} f(x_1,x_2) d\mu_1(x_1) \right) d\mu_2(x_2). 
		\end{align*}
	\end{tw}
	Powyższe twierdzenie uogólnia się w sposób bezpośredni na wielowymiarowe przestrzenie produktowe.