\chapter{Przeliczalność i nieprzeliczalność}
\section{Konstrukcja von Neumanna liczb naturalnych}
	\begin{axm}[Aksjomat nieskończoności]
		Istnieje zbiór $X$ o następujących właściwościach:
		\begin{itemize}
			\item $\emptyset \in X$
			\item $\forall_{y \in X} \left( S(y) \in X\right)$
		\end{itemize}
		gdzie $S(y)$ jest następnikiem porządkowym zbioru $y$, tj. $S(y) = y \cup \{y\}$. Zbiór spełniający powyższe właściwości nazywać będziemy \textit{zbiorem induktywnym}.
	\end{axm}
	\begin{lem}
		Niech $\mathcal{P} = \{ Y \subset X : Y \; \text{jest zbiorem induktywnym}\}$, wtedy zbiór
		$\cap_{P \in \mathcal{P}} P$ jest zbiorem induktywnym.
	\end{lem}
	\begin{tw}
		Istnieje najmniejszy, pod względem inkluzji, zbiór induktywny.
	\end{tw}
	\begin{df}
		Najmniejszy pod względem inkluzji zbiór induktywny nazywamy \textit{zbiorem liczb naturalnych} i oznaczamy przez $\mathbb{N}$.
		Korzystając z induktywności $\mathbb{N}$:
		\begin{itemize}
			\item $\emptyset \in \mathbb{N}$ oznaczamy jako $0$,
			\item $S(\emptyset) = \{\emptyset\}$ oznaczamy jako $1$,
			\item $S(\{\emptyset\}) = \{\emptyset, \{\emptyset\} \}$ oznaczamy jako $2$,
			\item i tak dalej $\ldots$
		\end{itemize}
		Elementy tego zbioru nazywamy \textit{liczbami naturalnymi}.
	\end{df}

	\section{Teoria mocy}
	\subsection{Zbiory przeczliczalne}
	\begin{df}
		Zbiory $A$ i $B$ nazywamy \textit{równolicznymi}, gdy istnieje bijekcja $f:A\rightarrow B$. Równoliczność zbiorów oznaczamy przez $A \sim B$.
	\end{df}
	
	\begin{tw}\label{rownolicznosc}
		Dla dowolnych zbiorów $X,Y,Z$ zachodzą następujące wzory:
		\begin{enumerate}
			\item $X \sim X$,
			\item $X \sim Y \implies Y \sim X$,
			\item $(X \sim Y) \wedge (Y \sim Z) \implies X \sim Z$.
		\end{enumerate}
	\end{tw}	

	\begin{uwg}
		Z twierdzenia \ref{rownolicznosc} wynika, iż relacja równoliczności jest relacją równoważności.
	\end{uwg}
	
	\begin{df}
		Zbiór $A$ nazywamy \textit{skończonym}, gdy $A \sim n$, dla $n \in \mathbb{N}$.
	\end{df}
	
	\begin{df}
		Zbiór $A$ nazywamy \textit{nieskończonym}, gdy $A$ nie jest skończony.
	\end{df}
	
	\begin{df}
		\textit{Zbiorami przeliczalnymi} nazywamy zbiory skończone lub równoliczne ze zbiorem $\mathbb{N}$ wszystkich liczb naturalnych.
	\end{df}
	
	\begin{lem}
		Własności zbiorów przeliczalnych:
		\begin{itemize}
			\item podzbiór przeliczalnego zbioru jest przeliczalny,
			\item suma dowolnej skończonej ilości zbiorów przeliczalnych, jest zbiorem przeliczalnym,
			\item iloczyn kartezjański skończonej ilości zbiorów przeliczalnych jest przeliczalny.
			\item dla każdej przeliczalnej rodziny indeksowanej zbiorów $(A_n)_{n \in \mathbb{N}}$, takiej, że $A_n$ jest zbiorem przeliczalnym dla każdego $n \in \mathbb{N}$, suma uogólniona $\cup_{n \in \mathbb{N}} A_n$ jest zbiorem przeliczalnym.
			\item zbiór wszystkich ciągów skończonych o wyrazach należących do ustalonego zbioru przeliczalnego jest zbiorem przeliczalnym.
		\end{itemize}	
	\end{lem}
	
	\begin{przyk}
		Zbiór wszystkich liczb całkowitych ujemnych $\mathbb{Z}^-$ jest zbiorem przeliczalnym, ponieważ istnieje bijekcja $f:\mathbb{N} \to \mathbb{Z}^-$ określona wzorem $f(n) = -n$.
	\end{przyk}
	
	\begin{przyk}
		Zbiór wszystkich liczb całkowitych $\mathbb{Z}$, jako suma skończonej ilości zbiorów przeliczalnych $\mathbb{Z}^- \cup \{0\} \cup \mathbb{Z}^+$, jest zbiorem przeliczalnym.
	\end{przyk}
	
	\begin{przyk}
		Zbiór liczb pierwszych, jako podzbiór zbioru liczb naturalnych, jest przeliczany.
	\end{przyk}
	
	\begin{przyk}
		Zbiór $\mathbb{Q}$ wszystkich liczb wymiernych jest zbiorem przeliczalnym.
		\begin{proof}
			Niech $\mathbb{Z}^{*} = \mathbb{Z}/\{0\}$.
			Zbiór $\mathbb{Z} \times \mathbb{Z}^*$, jako iloczyn kartezjański zbiorów przeliczalnych, jest zbiorem przeliczalnym. Istnieje zatem bijekcja $f:\mathbb{N} \to \mathbb{Z} \times \mathbb{Z}^*$. \\
			Każda liczba wymierna da się przedstawić w postaci $\frac{m}{n}$, gdzie $m \in \mathbb{Z}, \; m \in \mathbb{Z}^*$. Niech funkcja $g:\mathbb{Z} \times \mathbb{Z}^* \to \mathbb{Q}$ dana będzie następującym wzorem: $g(m,n) = \frac{m}{n}$ dla każdego elementu $(m,n) \in \mathbb{Z} \times \mathbb{Z}^*$. \\
			Niech $h= g \circ f$. Wówczas  $h: \mathbb{N} \to  \mathbb{Q}$, jako złożenie dwóch bijekcji, jest bijekcją. Wnioskujemy stąd, że zbiór liczb wymiernych jest przeliczalny.
		\end{proof}		 
	\end{przyk}
	
	
	\subsection{Zbiory nieprzeliczalne}
	\begin{df}
		Zbiór nazywamy \textit{nieprzeliczalnym}, gdy nie jest przeliczalny.
	\end{df}		
	
	\begin{tw}[Cantora]\label{tw.Cantora}
		Zbiór liczb rzeczywistych nie jest przeliczalny.
	\end{tw}	
	
	\begin{przyk}
		Zbiór $2^{\mathbb{N}}$ jest zbiorem nieprzeliczalnym.
	\end{przyk}
	
	\begin{przyk}
		Zbiór $\mathbb{N}^{\mathbb{N}}$ jest zbiorem nieprzeliczalnym.
	\end{przyk}
	
	\begin{przyk}
		Zbiór $\mathbb{Q}^* = \{ x \in \mathbb{R}: x \not \in \mathbb{Q} \}$ liczby niewymiernych jest nieprzeliczalny.
		\begin{proof}
			Gdyby zbiór $\mathbb{Q}^*$ był przeliczalny, to zbiór liczb rzeczywistych $\mathbb{R} = \mathbb{Q} \cup \mathbb{Q}^*$, jako suma dwóch zbiorów przeliczalnych, również byłby zbiorem przeliczalnym, co przeczy twierdzeniu \ref{tw.Cantora}.
		\end{proof}
	\end{przyk}

	\subsection{Zbiory mocy continuum}	
	\begin{df}
		Mówimy, że zbiór jest \textit{mocy continuum}, gdy jest równoliczny z  $\mathbb{R}$.
	\end{df}
	
	\begin{tw}
		Jeżeli $A \subset \mathbb{R}$ i $A$ zawiera pewien przedział otwarty, to $A$ jest mocy continuum.
	\end{tw}
	
	\begin{tw}
	Jeżeli $B \subset A$ jest przeliczalnym podzbiorem zbioru $A$ mocy continuum, to	$A \setminus B$ jest mocy continuum.
	\end{tw}
	
	\begin{tw}
		Jeżeli $B$ jest przeliczalnym, a  $A$ jest mocy continuum, to $A \cup B$ jest mocy continuum.
	\end{tw}

	\begin{przyk}
		Zbiór wszystkich funkcji $f: \mathbb{N} \to \{0,1\}$, tj. zbiór wszystkich ciągów $(a_n)_{n \in \mathbb{N}}$, takich że $a_n \in \{0,1\}$ dla każdego $n \in \mathbb{N}$, jest mocy continuum.
	\end{przyk}
	
	\begin{przyk}
		Zbiór $ 2^{\mathbb{N}}$ jest mocy continuum.
	\end{przyk}
	
	\begin{przyk}
		Każdy przedział otwarty $(a, b) \subset \mathbb{R}$, gdzie $a < b$, jest mocy continuum.
	\end{przyk}
	
	\begin{przyk}
		Przedział otwarty $(-\frac{1}{2}\pi, \; \frac{1}{2}\pi) \subset \mathbb{R}$ jest mocy continuum. 				
		\begin{proof}
			Funkcja $f:(-\frac{1}{2}\pi, \; \frac{1}{2}\pi) \to \mathbb{R}$ określona wzorem $f(x) = \tg(x)$ jest różnowartościowa i~przekształca swoją dziedzinę na $\mathbb{R}$, ustala więc równoliczność tych zbiorów.
		\end{proof}
	\end{przyk}
	
	\section{Nierówności dla liczb kardynalnych}
	\begin{df}
		Niech $|A|= n$ oraz $|B| = m$. Przyjmujemy, że liczba kardynalna $n$ jest nie większa od liczby kardynalnej $m$, wtedy i tylko wtedy, kiedy istnieje iniekcja $f:A \to B$, co oznaczamy poprzez $A <_m B$.
	\end{df}
	
	\begin{df}
	 Jeżeli $A  \leq_m  B$ i nieprawda, że $A  \sim_m  B$, to mówimy, że liczba kardynalna zbioru $A$ jest \textit{mniejsza} od liczby kardynalnej zbioru $B$, co oznaczamy poprzez $A <_m B$.
	\end{df}
	
	\begin{tw}
		Następujące warunki są równoważne:
		\begin{itemize}
			\item	Dla dowolnych zbiorów $A,B$ zachodzi $A  \leq_m  B$ i $B  \leq_m  A$, to $ A  \sim_m  B$.
			\item	Dla dowolnych zbiorów $A,B$ zachodzi $A  \leq_m  B$ i $B \subset  A$, to $ A  \sim_m  B$.
			\item	Dla dowolnych zbiorów $A,B,C$ zachodzi $A  <_m  B$ i  $B <_m C$, to $A  <_m  C$.
		\end{itemize}
	\end{tw}
	
	\begin{tw}[Cantora - Bernsteina]
		Jeżeli $A  \leq_m  B$ i $B  \leq_m  A$ to $A  \sim_m  B$.
	\end{tw}
	
	\begin{tw}[Cantora]
		$A  <_m  \mathcal{P} (A)$.
	\end{tw}
	
	\begin{tw}
		Nie istnieje zbiór wszystkich zbiorów.
	\end{tw}
	
	\begin{tw}
		$\mathbb{N}  <_m   \mathcal{P} (\mathbb{N})  \sim_m  2^{\mathbb{N}}  \sim_m  \mathbb{R}$.
	\end{tw}
	\textbf{Hipoteza continuum}: czy istnieje taki zbiór, którego moc dałoby się ulokować pomiędzy mocą zbioru liczb naturalnych a mocą continuum. Czyli, czy istnieje $A$ takie, że $\mathbb{N}  <_m  A  <_m  \mathbb{R}$?